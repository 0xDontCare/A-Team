\chapter{Arhitektura i dizajn sustava}
		
		\textbf{\textit{dio 1. revizije}}\\

		Arhitektura koju koristimo se zasniva na MVC (Model-View-Controller) konceptu, varijacije arhitekture zasnovane na događajima.
        \\
        \\
        Cjelokupni sustav se može podijeliti na četiri glavne komponente:
            \begin{itemize}
                \item Web preglednik
                \item Web poslužitelj (server)
                \item Web aplikaciju
                \item Bazu podataka
            \end{itemize}
    
        \begin{figure}[h]
            \centering
            \includegraphics[width=0.5\textwidth]{slike/mvc.png}
            \caption{MVC model}
            \label{fig:mesh1}
        \end{figure}
        
        \textbf{Web preglednik} je program (software) koji omogućava korisnicima pregledavanje i prikazivanje web stranica na World Wide Webu (WWW). Predstavlja sučelje između korisnika i internetskog sadržaja. Pomoću njega korisnik sustava komunicira s web aplikacijom, točnije \textit{View} i \textit{Controller} komponentom.
        
        \pagebreak

        \textbf{Web aplikaciju} pokreće \textbf{poslužitelj}. To je program čiji je osnovni zadatak odgovarati na HTTP (\textit{Hyper Text Transfer Protocol}) zahtjeve klijenata, primati i slati određene resurse - posluživati samu aplikaciju. 
        
        \begingroup 
        U većini slučajeva klijent će zahtjevati pristup podacima kojima web poslužitelj nema pristup. U tom slučaju će \textit{Controller} komponenta poslati zahtjev \textit{Model} komponenti, zaslužnoj za pohranu korisničkih podataka, koju u našem slučaju predstavlja \textbf{baza podataka}. 
        \endgroup
        
        \begingroup
        Baza podataka organizirana je zbirka logički povezanih, pretražljivih i međusobno ovisnih podataka. Ključna je komponenta mnogih aplikacija i sustava zbog mogućnosti sigurne pohrane i brzog pretraživanja SQL upitima.
	  \endgroup

        \begingroup
        Za našu implementaciju korisničkog sučelja odabrali smo programski jezik \textit{JavaScript} s bibliotekom \textit{React}. Za implementaciju poslužiteljske strane odabrali smo programski jezik \textit{Java} i \textit{Spring} framework, točnije proširenje \textit{Spring Boot}. Kao razvojnu okolinu odabrali smo \textit{Visual Studio Code} i \textit{JetBrains IntelliJ IDEA}. Za implementaciju baze podataka odabrali smo \textit{PostgreSQL}.
        \endgroup

		\begin{figure}[h]
            \centering
            \includegraphics[width=1\textwidth]{slike/client-server-db.png}
            \caption{Prikaz osnovnog rada sustava}
            \label{fig:mesh1}
        \end{figure}

        \pagebreak
				
		\section{Baza podataka}
			
		\noindent Za naš sustav koristit ćemo relacijsku bazu podataka, koja je strukturno pogodna za modeliranje stvarnog svijeta. Osnovna komponenta ove baze je relacija, koja se definira svojim imenom i skupom atributa. Glavna svrha ove baze podataka je omogućiti brzo i jednostavno pohranjivanje, mijenjanje i dohvaćanje podataka radi daljnje obrade. Baza podataka za ovu aplikaciju uključuje sljedeće entitete:
		
		\begin{packed_item}
			
			\item Korisnik
			\item Registrirani			
			\item Skloniste
			\item Oglas
			\item Ljubimac
			\item Lokacija
			\item Mjesto
			\item Zupanija
			\item Slika
			\item Poruka
			\item Kategorija
			
		\end{packed_item}
		
			\subsection{Opis tablica}
			

				\noindent \textbf{Korisnik} Entitet sadržava sve važne informacije o korisniku web-aplikacije. Atributi su sljedeći: korisnickoIme, lozinka, email, telefon, postBr. Primarni ključ ovog entiteta je atribut korisnickoIme. Entitet je povezan s entitetom Mjesto pomoću atributa postBr kroz vezu \textit{Many-to-One}, a s entitetima Registrirani i Skloniste, kao specijalizacije, pomoću atributa korisnickoIme. Postoji i povezanost vezom \textit{One-to-Many} s identifikacijskim slabim entitetom Poruka pomoću atributa korisnickoIme.
				
				
				\begin{longtblr}[
					label=none,
					entry=none
					]{
						width = \textwidth,
						colspec={|X[6,l]|X[6, l]|X[20, l]|}, 
						rowhead = 1,
					} %definicija širine tablice, širine stupaca, poravnanje i broja redaka naslova tablice
					\hline \SetCell[c=3]{c}{\textbf{Korisnik }}	 \\ \hline[3pt]
					\SetCell{LightGreen}korisnickoIme & VARCHAR	&  	jedinstveno korisničko ime  	\\ \hline
					lozinka	& VARCHAR & hash lozinke  	\\ \hline 
					email & VARCHAR & e-mail adresa korisnika  \\ \hline 
					telefon & VARCHAR	& telefonski broj korisnika 		\\ \hline 
					\SetCell{LightBlue} postBr	& INT & poštanski broj mjesta stanovanja korisnika  	\\ \hline 
				\end{longtblr}
				
				
				\noindent \textbf{Registrirani} Entitet je specijalizacija entiteta Korisnik i sadrži dodatne podatke o registriranom korisniku. Atributi entiteta su ime, prezime i korisnickoIme. Entitet je kao specijalizacija povezan entitetom Korisnik kroz atribut korisnickoIme.
				
				
				\begin{longtblr}[
					label=none,
					entry=none
					]{
						width = \textwidth,
						colspec={|X[6,l]|X[6, l]|X[20, l]|}, 
						rowhead = 1,
					} %definicija širine tablice, širine stupaca, poravnanje i broja redaka naslova tablice
					\hline \SetCell[c=3]{c}{\textbf{Registrirani }}	 \\ \hline[3pt]
					ime	& VARCHAR & ime registriranog korisnika  	\\ \hline 
					prezime & VARCHAR & prezime registriranog korisnika  \\ \hline 
					\SetCell{LightBlue} korisnickoIme	& VARCHAR & jedinstveno korisničko ime	\\ \hline 
				\end{longtblr}
				
				
				\noindent \textbf{Skloniste} Entitet je specijalizacija entiteta Korisnik i sadrži dodatne podatke o skloništu za životinje. Atributi entiteta su nazivSklonista i korisnickoIme. Entitet je kao specijalizacija povezan entitetom Korisnik kroz atribut korisnickoIme.
				
				
				\begin{longtblr}[
					label=none,
					entry=none
					]{
						width = \textwidth,
						colspec={|X[6,l]|X[6, l]|X[20, l]|}, 
						rowhead = 1,
					} %definicija širine tablice, širine stupaca, poravnanje i broja redaka naslova tablice
					\hline \SetCell[c=3]{c}{\textbf{Skloniste }}	 \\ \hline[3pt]
					nazivSklonista & VARCHAR	&  	naziv skloništa za životinje  	\\ \hline
					\SetCell{LightBlue}korisnickoIme	& VARCHAR & jedinstveno korisničko ime  	\\ \hline
				\end{longtblr}
				
				
				\noindent \textbf{Ljubimac} Entitet sadržava sve važne podatke o kućnome ljubimcu. Atributi entiteta su: sifLjubimac, ime, vrsta, starost, boja, opis. Primarni ključ ovog entiteta je atribut sifLjubimac. 
				Entitet je povezan vezom \textit{One-to-Many} kroz atribut sifLjubimac s entitetom Slika. Također, postoji veza \textit{One-to-Many} s entitetom Oglas pomoću atributa sifOglas.
				
				\begin{longtblr}[
					label=none,
					entry=none
					]{
						width = \textwidth,
						colspec={|X[6,l]|X[6, l]|X[20, l]|}, 
						rowhead = 1,
					} %definicija širine tablice, širine stupaca, poravnanje i broja redaka naslova tablice
					\hline \SetCell[c=3]{c}{\textbf{Ljubimac }}	 \\ \hline[3pt]
					\SetCell{LightGreen}sifLjubimac & INT	&  	jedinstvena šifra ljubimca  	\\ \hline
					ime	& VARCHAR & ime na koje se ljubimac odaziva  	\\ \hline 
					vrsta & VARCHAR & vrsta ljubimca  \\ \hline 
					starost & INT	& starost ljubimca 		\\ \hline 
					boja & VARCHAR & boja ljubimca  \\ \hline 
					opis & VARCHAR	& dodatan opis ljubimca 		\\ \hline 
				\end{longtblr}
				
				
				\noindent \textbf{Oglas} Entitet sadržava sve važne informacije o oglasu za kućnog ljubimca. Atributi su sljedeći: sifOglas, datumVrijemeNestanka, koordinate, sifLjubimac, sifKategorija, korisnickoIme. Primarni ključ ovog entiteta je atribut sifOglas. Entitet je u vezi \textit{Many-to-One} s entitetom Lokacija pomoću atributa koordinate, u vezi \textit{Many-to-One} s entitetom Ljubimac pomoću atributa sifLjubimac, u vezi \textit{Many-to-One} s entitetom Kategorija pomoću atributa sifKategorija te u vezi \textit{Many-to-One} s entitetom Korisnik pomoću atributa korisnickoIme.
				
				
				\begin{longtblr}[
					label=none,
					entry=none
					]{
						width = \textwidth,
						colspec={|X[10,l]|X[6, l]|X[16, l]|}, 
						rowhead = 1,
					} %definicija širine tablice, širine stupaca, poravnanje i broja redaka naslova tablice
					\hline \SetCell[c=3]{c}{\textbf{Oglas }}	 \\ \hline[3pt]
					\SetCell{LightGreen}sifOglas & INT	&  	jedinstveni identifikator oglas  	\\ \hline
					datumVrijemeNestanka	& TIMESTAMP & datum i vrijeme nestanka ljubimca  	\\ \hline 
					\SetCell{LightBlue}koordinate & VARCHAR & koordinate lokacije nestanka ljubimca  \\ \hline 
					\SetCell{LightBlue}sifLjubimac & INT	& jedinstveni identifikator ljubimca 		\\ \hline 
					\SetCell{LightBlue} sifKategorija	& INT & jedinstveni identifikator kategorije oglasa  	\\ \hline
					\SetCell{LightBlue}korisnickoIme & VARCHAR & jedinstveno korisničko ime korisnika koji je postavio oglas \\ \hline
					 
				\end{longtblr}
				
				
				\noindent \textbf{Lokacija} Entitet sadržava podatke o lokaciji. Atributi entiteta su koordinate i postBr, gdje je primarni ključ atribut koordinate. Postoji veza \textit{Many-to-One} s entitetom Mjesto kroz atribut postBr.
				
				
				\begin{longtblr}[
					label=none,
					entry=none
					]{
						width = \textwidth,
						colspec={|X[6,l]|X[6, l]|X[20, l]|}, 
						rowhead = 1,
					} %definicija širine tablice, širine stupaca, poravnanje i broja redaka naslova tablice
					\hline \SetCell[c=3]{c}{\textbf{Lokacija }}	 \\ \hline[3pt]
					\SetCell{LightGreen}koordinate & VARCHAR	&  	jedinstvene koordinate lokacije  	\\ \hline
					\SetCell{LightBlue} postBr	& INT & poštanski broj mjesta	\\ \hline 
				\end{longtblr}
				
				
				\noindent \textbf{Mjesto} Entitet je sastavljen od podataka vezanih za mjesto pomoću atributa postBr, koji je primarni ključ, nazivMjesto i sifZup. Postoje veze \textit{One-to-Many} pomoću atributa postBr s entitetima Lokacija i Korisnik te veza \textit{Many-to-One} s entitetom Zupanija pomoću atributa sifZup.
				
				
				\begin{longtblr}[
					label=none,
					entry=none
					]{
						width = \textwidth,
						colspec={|X[6,l]|X[6, l]|X[20, l]|}, 
						rowhead = 1,
					} %definicija širine tablice, širine stupaca, poravnanje i broja redaka naslova tablice
					\hline \SetCell[c=3]{c}{\textbf{Mjesto }}	 \\ \hline[3pt]
					\SetCell{LightGreen}postBr & INT	&  	poštanski broj mjesta	\\ \hline
					nazivMjesto	& VARCHAR & naziv mjesta  	\\ \hline 
					\SetCell{LightBlue} sifZup	& INT & jedinstveni identifikator županije	\\ \hline 
				\end{longtblr}
				
				
				\noindent \textbf{Zupanija} Entitet je sastavljen od podataka vezanih za županiju pomoću atributa sifZup i nazivZup. Primarni ključ je sifZup. Postoji veza \textit{One-to-Many} pomoću atributa sifZup s entitetom Mjesto.
				
				
				\begin{longtblr}[
					label=none,
					entry=none
					]{
						width = \textwidth,
						colspec={|X[6,l]|X[6, l]|X[20, l]|}, 
						rowhead = 1,
					} %definicija širine tablice, širine stupaca, poravnanje i broja redaka naslova tablice
					\hline \SetCell[c=3]{c}{\textbf{Zupanija }}	 \\ \hline[3pt]
					\SetCell{LightGreen}sifZup & INT	&  	jedinstveni identifikator županije  	\\ \hline
					nazivZup	& VARCHAR & naziv županije  	\\ \hline 
				\end{longtblr}
				
				
				\noindent \textbf{Poruka} Entitet sadržava sve važne informacije o poruci te je oblikovan kao identifikacijski slabi entitet. Atributi su sljedeći: tekstPoruke, datumVrijemeSlanja, korisnickoIme, koordinate, vezaNaSliku i sifOglas.  Entitet je povezan s entitetom Korisnik pomoću atributa korisnickoIme kroz vezu \textit{Many-to-One}, s entitetom Lokacija kroz vezu \textit{Many-to-One} pomoću atributa koordinate te s entitetom Oglas kroz vezu \textit{Many-to-One} pomoću atributa sifOglas.
				
				
				\begin{longtblr}[
					label=none,
					entry=none
					]{
						width = \textwidth,
						colspec={|X[9,l]|X[6, l]|X[17, l]|}, 
						rowhead = 1,
					} %definicija širine tablice, širine stupaca, poravnanje i broja redaka naslova tablice
					\hline \SetCell[c=3]{c}{\textbf{Poruka }}	 \\ \hline[3pt]
					tekstPoruke & VARCHAR & tekst poruke \\ \hline
					\SetCell{LightBlue}datumVrijemeSlanja & TIMESTAMP	&  	datum i vrijeme slanja poruke  	\\ \hline
					\SetCell{LightBlue}korisnickoIme	& VARCHAR & jedinstveno korisnicko ime  	\\ \hline 
					\SetCell{LightBlue}koordinate & VARCHAR & jedinstvene koordinate lokacije  \\ \hline 
					\SetCell{LightBlue}vezaNaSliku & VARCHAR	& jedinstvena poveznica do slike 		\\ \hline 
					\SetCell{LightBlue}sifOglas & INT	& jedinstveni identifikator oglasa 		\\ \hline 
				\end{longtblr}
				
				
				\noindent \textbf{Slika} Entitet sadržava podatke o slici. Atributi su vezaNaSliku, koji je primarni ključ, i sifLjubimac. S entitetom su povezani: Ljubimac s vezom \textit{Many-to-One} pomoću atributa sifLjubimac, Poruka s vezom \textit{One-to-Many} uz pomoć atributa vezaNaSliku.
				
				
				\begin{longtblr}[
					label=none,
					entry=none
					]{
						width = \textwidth,
						colspec={|X[6,l]|X[6, l]|X[20, l]|}, 
						rowhead = 1,
					} %definicija širine tablice, širine stupaca, poravnanje i broja redaka naslova tablice
					\hline \SetCell[c=3]{c}{\textbf{Slika }}	 \\ \hline[3pt]
					\SetCell{LightGreen}vezaNaSliku & VARCHAR	&  	jedinstvena poveznica do slike  	\\ \hline
					\SetCell{LightBlue}sifLjubimac	& INT & jedinstveni identifikator ljubimca  	\\ \hline  
				\end{longtblr}
				
				
				\noindent \textbf{Kategorija} Entitet se sastoji od podataka o kategoriji: sifKategorija (primarni ključ) i opisKategorija. Postoji veza \textit{One-to-Many} između entiteta Kategorija i Oglasa kroz atribut sifKategorija.
				
				
				\begin{longtblr}[
					label=none,
					entry=none
					]{
						width = \textwidth,
						colspec={|X[6,l]|X[6, l]|X[20, l]|}, 
						rowhead = 1,
					} %definicija širine tablice, širine stupaca, poravnanje i broja redaka naslova tablice
					\hline \SetCell[c=3]{c}{\textbf{Kategorija }}	 \\ \hline[3pt]
					\SetCell{LightGreen}sifKategorija & INT	&  	jedinstveni identifikator kategorije  	\\ \hline
					opisKategorija	& VARCHAR & opis kategorije  	\\ \hline 
				\end{longtblr}
				
			
			\subsection{Dijagram baze podataka}
				\begin{figure}[h]
					\centering
					\includegraphics[width=1\textwidth]{slike/ERdiagram.png}
					\caption{E-R dijagram baze podataka}
					\label{fig:mesh1}
				\end{figure}
			
			\eject
			
			
		\section{Dijagram razreda}
		
			\textit{Potrebno je priložiti dijagram razreda s pripadajućim opisom. Zbog preglednosti je moguće dijagram razlomiti na više njih, ali moraju biti grupirani prema sličnim razinama apstrakcije i srodnim funkcionalnostima.}\\
			
			\textbf{\textit{dio 1. revizije}}\\
			
			\textit{Prilikom prve predaje projekta, potrebno je priložiti potpuno razrađen dijagram razreda vezan uz \textbf{generičku funkcionalnost} sustava. Ostale funkcionalnosti trebaju biti idejno razrađene u dijagramu sa sljedećim komponentama: nazivi razreda, nazivi metoda i vrste pristupa metodama (npr. javni, zaštićeni), nazivi atributa razreda, veze i odnosi između razreda.}\\
			
			\textbf{\textit{dio 2. revizije}}\\			
			
			\textit{Prilikom druge predaje projekta dijagram razreda i opisi moraju odgovarati stvarnom stanju implementacije}
			
			
			
			\eject
		
		\section{Dijagram stanja}
			
			
			\textbf{\textit{dio 2. revizije}}\\
			
			\textit{Potrebno je priložiti dijagram stanja i opisati ga. Dovoljan je jedan dijagram stanja koji prikazuje \textbf{značajan dio funkcionalnosti} sustava. Na primjer, stanja korisničkog sučelja i tijek korištenja neke ključne funkcionalnosti jesu značajan dio sustava, a registracija i prijava nisu. }
			
			
			\eject 
		
		\section{Dijagram aktivnosti}
			
			\textbf{\textit{dio 2. revizije}}\\
			
			 \textit{Potrebno je priložiti dijagram aktivnosti s pripadajućim opisom. Dijagram aktivnosti treba prikazivati značajan dio sustava.}
			
			\eject
		\section{Dijagram komponenti}
		
			\textbf{\textit{dio 2. revizije}}\\
		
			 \textit{Potrebno je priložiti dijagram komponenti s pripadajućim opisom. Dijagram komponenti treba prikazivati strukturu cijele aplikacije.}
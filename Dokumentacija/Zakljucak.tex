\chapter{Zaključak i budući rad}

Zadatak naše grupe bio je razvoj web aplikacije za pronalazak nestalih ljubimaca gdje postoji mogućnost objavljivanja, izmjene i brisanja oglasa te razgovora putem chat-a s drugim sudionicima koji su uključeni u potragu za nestalim ljubimcima. Nakon 15 tjedana rada u timu i razvoja, ostvarili smo zadani cilj. Sama provedba projekta bila je kroz dvije faze.

Prva faza projekta uključivala je okupljanje tima za razvoj aplikacije, dodjelu projektnog zadatka i intenzivan rad na dokumentiranju zahtjeva. Kvaliteta provedbe prve faze uvelike je olakšala daljnji rad pri realizaciji osmišljenog sustava. Izrađeni obrasci i dijagrami(obrasci uporabe, sekvencijski dijagrami, model baze podataka, dijagram razreda) bili su od pomoći podtimovima zaduženima za razvoj \textit{backenda} i \textit{frontenda}. Izrada vizualnih prikaza idejnih rješenja problemskih zadataka uštedjela je mnogo vremena u drugom ciklusu kada su članovi tima nailazili na nedoumice oko implementacije rješenja.

Druga faza projekta, iako nešto kraća od prve, bila je puno intenzivnija po pitanju samostalnog rada članova. Manjak iskustva članova u izradi sličnih implementacijskih rješenja primorao je članove na samostalno učenje odabranih alata i programskih jezika kako bi ispunili dogovorene ciljeve. Osim realizacije rješenja, u drugoj fazi je bilo potrebno dokumentirati ostale UML dijagrame i izraditi popratnu dokumentaciju kako bi budući korisnici mogli lakše koristiti ili vršiti preinake na sustavu. Dobro izrađen kostur projekta uštedio nam je mnogo vremena prilikom izrade aplikacije te smo izbjegli moguće pogreške u izradi koje bi bile vremenski skupe za ispravljanje u daljnjoj fazi projekta.

Komunikacija među članovima tima bila je putem Discorda čime smo postigli informiranost svih članova grupe o napretku projekta. Moguće proširenje postojeće inačice sustava je izrada mobilne aplikacije čime bi cilj projektnog zadatka bio ostvaren u većoj mjeri.

Sudjelovanje na ovakvom projektu bio je vrijedno iskustvo svim članovima tima jer smo kroz intenzivnih nekoliko tjedana rada iskusili zajednički rad na istom projektu. Također, osjetili smo važnost dobre vremenske organiziranosti i koordiniranosti između članova tima. Zadovoljni smo postignutim bez obzira na golemi prostor za usavršavanje izrađene aplikacije što je posljedica neiskustva na takvim i sličnim projektima.

Jedina funkcionalnost koja nije implementirana je dodavanje slika uz poruku u chatu. Ova funkcionalnost nije implementirana zbog nedostatka vremena, ali je ostavljena kao mogućnost za budući rad.

\eject